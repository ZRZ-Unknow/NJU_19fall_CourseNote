\documentclass[a4paper]{article}
% \usepackage[margin=1.25in]{geometry}
\usepackage[inner=2.0cm,outer=2.0cm,top=2.5cm,bottom=2.5cm]{geometry}
% \usepackage{ctex}
\usepackage{color}
\usepackage{graphicx}
\usepackage{amssymb}
\usepackage{amsmath}
\usepackage{amsthm}
\usepackage{bm}
\usepackage{hyperref}
\usepackage{multirow}
\usepackage{enumerate}

\newcommand{\homework}[5]{
    \pagestyle{myheadings}
    \thispagestyle{plain}
    \newpage
    \setcounter{page}{1}
    \noindent
    \begin{center}
    \framebox{
        \vbox{\vspace{2mm}
        \hbox to 6.28in { {\bf Optimization Methods \hfill #2} }
        \vspace{6mm}
        \hbox to 6.28in { {\Large \hfill #1 \hfill} }
        \vspace{6mm}
        \hbox to 6.28in { {\it Instructor: {\rm #3} \hfill Name: {\rm #4}, StudentId: {\rm #5}}}
        \vspace{2mm}}
    }
    \end{center}
    % \markboth{#4 -- #1}{#4 -- #1}
    \vspace*{4mm}
}


\newenvironment{solution}
{\color{blue} \paragraph{Solution.}}
{\newline \qed}

\begin{document}
%==========================Put your name and id here==========================
\homework{Homework 2}{Fall 2019}{Lijun Zhang}{Renzhe Zhou}{181220076}

\paragraph{Notice}
\begin{itemize}
    \item The submission email is: \textbf{njuoptfall2019@163.com}.
    \item Please use the provided \LaTeX{} file as a template. If you are not familiar with \LaTeX{}, you can also use Word to generate a \textbf{PDF} file.
\end{itemize}

\paragraph{Problem 1: First-order Convexity Condition}
~\\
If $f$ is a continuous function on some interval $\mathbf{I}$,
\begin{enumerate}[a)]
    \item Prove that $f$ is a convex function if and only if $\forall x_1,x_2\in\mathbf{I}$,
    \begin{equation}\label{equa:1}
    f\left(\frac{x_1+x_2}{2}\right)\leq \frac{1}{2}[f(x_1)+f(x_2)].
    \end{equation}
    \item Prove that $f(x)=e^x$ is a convex function.
    \item If $m,n>0,p>1$ and $1/p+1/q=1$, prove that $mn\leq\frac{m^p}{p}+\frac{n^q}{q}.$
\end{enumerate}

\begin{solution}
\begin{enumerate}[a)]
  \item The necessity is obviously: take $\theta=1/2$, we have if $f$ is convex then $f(\frac{x_1+x_2}{2})\leq \frac{1}{2}[f(x_1)+f(x_2)]$.\\
      The sufficiency part is as following:\\
      Firstly let's prove that any point in $[x,y]$ which is $\lambda x+(1-\lambda)y$ $(0\le\lambda\le1)$ can be expressed as $\frac{m}{2^k}x+(1-\frac{m}{2^k})y$. Using half-approximation, $\lambda x+(1-\lambda)y$ is in interval $\frac{x+y}{2}$ and $\{[\frac{x+y}{2}-\lambda x+(1-\lambda)y]>0?x:y\}$, so continuously use half-approximation we can get $\lambda x+(1-\lambda)y+\epsilon=\frac{m}{2^k}x+(1-\frac{m}{2^k})y$ when $\epsilon\to0$ and $k\to\infty$, in which $m/2^k=\frac{\lfloor2^k\lambda\rfloor}{2^k}$. So when $k\to\infty$, $m/2^k\to\lambda$.\\
      We have $f(\frac{3}{4}x+\frac{1}{4}y)=f(\frac{1}{2}(\frac{1}{2}(x+y)+x))\le\frac{1}{2}(f(\frac{1}{2}(x+y))+f(x))\le\frac{3}{4}f(x)+\frac{1}{4}f(y)$ because of $f(\frac{x+y}{2})\leq \frac{1}{2}[f(x)+f(y)]$,
      repeating this process we can get $f(\frac{m}{2^k}x+(1-\frac{m}{2^k})y)\le\frac{m}{2^k}f(x)+(1-\frac{m}{2^k})f(y)$, where $m\in\{2^i|i=0,1,2,\cdots,k\}$. So because f is continuous, then $\lim_{\epsilon\to0}f(x+\epsilon)=f(x)$ which means $f(\frac{m}{2^k}x+(1-\frac{m}{2^k})y)=f(\lambda x+(1-\lambda)y),\frac{m}{2^k}f(x)+(1-\frac{m}{2^k})f(y)=\lambda f(x)+(1-\lambda)f(y)$, therefore, $f(\lambda x+(1-\lambda)y\le\lambda f(x)+(1-\lambda)f(y)$, so $f$ is convex.
  \item Because of inequality of arithmetic and geometric means, $e^\frac{x_1+x_2}{2}=\sqrt{e^{x_1}e^{x_2}}\le\frac{1}{2}(e^{x_1}+e^{x_2})$, which means $f(\frac{x_1+x_2}{2})\leq \frac{1}{2}[f(x_1)+f(x_2)]$, through the conclusion from a), we can prove that $f(x)=e^x$ is a convex function.
  \item Because $p>1$,so $q=\frac{1}{1-\frac{1}{p}}>1$ too. Take $g(x)=\ln x$, $g''(x)=-x^{-2}<0$, so $g(x)$ is a concave function. Because $1/p+1/q=1$, we have $g(\frac{1}{p}m^p+\frac{1}{q}n^q)\ge\frac{1}{p}g(m^p)+\frac{1}{q}g(n^q)$, which is $\ln(\frac{m^p}{p}+\frac{n^q}{q})\ge\frac{1}{p}\ln(m^p)+\frac{1}{q}\ln(n^q)=\ln m+\ln n=\ln(mn)$. Because $g'(x)=1/x>0$, then we have $\frac{m^p}{p}+\frac{n^q}{q}\ge mn$.
\end{enumerate}
done.
\end{solution}

\paragraph{Problem 2: Second-order Convexity Condition}
~\\
Let $\mathcal{D}\subseteq\mathbf{R}^n$ be convex. For a function $f:\mathcal{D}\to\mathbf{R}$ and an $\alpha>0$, we say that $f$ is $\alpha$-exponentially concave, if $\exp(-\alpha f(x))$ is concave on $\mathcal{D}$. Suppose $f:\mathcal{D}\to\mathbf{R}$ is twice differentiable, give the necessary and sufficient condition of that $f$ is $\alpha$-exponentially concave and the detailed proof.
\begin{solution}
\begin{enumerate}[$\cdot$]
  \item $f$ is $\alpha$-exponentially concave if and only if $\alpha\nabla f(x)\nabla f(x)^T-\nabla f^2(x)\preceq 0$.
  \item Prove: Let's take $g(x)=e^{-\alpha x},h(x)=g(f(x))$, so $f$ is $\alpha$-exponentially concave if and only if $h(x)$ is a concave function if and only if $\nabla^2 h(x)\preceq 0$. From chain rules we have
      \begin{equation}\label{}
        \nabla h^2(x)=g'(f(x))\nabla^2f(x)+g''(f(x))\nabla f(x)\nabla f(x)^T
      \end{equation}
      and $g'(f(x))=-\alpha e^{-\alpha f(x)},g''(f(x))=\alpha^2 e^{-\alpha f(x)}$, so
      \begin{equation}\label{}
        \nabla h^2(x)=\alpha e^{-\alpha f(x)}[\alpha\nabla f(x)\nabla f(x)^T-\nabla^2f(x)]
      \end{equation}
      because $\alpha>0,e^{-\alpha f(x)}>0$, so $\nabla h^2(x)\preceq 0$ if and only if $\alpha\nabla f(x)\nabla f^T(x)-\nabla f^2(x)\preceq 0$. Therefore, $f$ is $\alpha$-exponentially concave if and only if $\alpha\nabla f(x)\nabla f(x)^T-\nabla f^2(x)\preceq 0$.
\end{enumerate}
done.
\end{solution}
\paragraph{Problem 3: Operations That Preserve Convexity}
~\\
Show that the following functions $f:\mathbf{R}^n\rightarrow\mathbf{R}$ are convex.
\begin{enumerate}[a)]
    \item $f(x)=\|{Ax-b}\|$, where $A\in \mathbf{R}^{m\times n}, b\in \mathbf{R}^{m}$ and $\|\cdot\|$ is a norm on $\mathbf{R}^{m}$.
    \item $f(x)=-(\textnormal{det}(A_0+x_1A_1+\cdots+x_nA_n))^{1/m}$, on $\{x|A_0+x_1A_1+\cdots+x_nA_n\succ 0\}$ where $A_i\in\mathbf{S}^m$.
    \item $f(x)=\mathbf{tr}((A_0+x_1A_1+\cdots+x_nA_n)^{-1})$, on $\{x|A_0+x_1A_1+\cdots+x_nA_n\succ 0\}$ where $A_i\in\mathbf{S}^m$.
\end{enumerate}
\begin{solution}
\begin{enumerate}[a)]
  \item Make g(x)=$\|x\|$,then $f(x)=g(Ax-b)$, and $g(x)$(norm) is convex, so after affine mapping $f(x)$ is convex too.
  \item Make $g(X)=-(det(X))^{1/m}$, then $f(x)=g(Ax^T+A_0)$ where $A=(A_1,A_2,\cdots,A_n),x=(x_1,x_2,\cdots,x_n)$, so $f$ is a composition and an affine transformation of $g$. Let's prove $g$ is convex so that $f$ is convex too. \\
      Let's transform $g$ into a line $h(t)=-(det(Z+tV))^{1/m}$ and prove h(t) is convex. \\
      $h(t)=-(det Z)^{1/m}(det(I+tZ^{-1/2}VZ^{-1/2}))^{1/m}
      =-(detZ)^{1/m}(\prod_{i=1}^m(1+t\lambda_i))^{1/m}$\\
      where $\lambda_i$ is the eigenvalues of $Z^{-1/2}VZ^{-1/2}$.\\
      So $h$ is a convex function of $t$ on $\{t|Z+tV\succ 0\}$ since $-(det(Z))^{1/m}<0$ and the geometric mean $(\prod_{i=1}^m(1+t\lambda_i))^{1/m}$ is concave. \\
      Above all, $f(x)$ is a convex function.
  \item Make $g(X)=tr(X)^{-1}$, then $f(x)=g(Ax^T+A_0)$ where $A=(A_1,A_2,\cdots,A_n),x=(x_1,x_2,\cdots,x_n)$, so $f$ is a composition and an affine transformation of $g$. Let's prove $g$ is convex so that $f$ is convex too. \\
      Let's transform $g$ into a line $h(t)=tr(Z+tV)^{-1}$ and prove h(t) is convex.\\
      $h(t)=tr( Z^{-1}(I+tZ^{-1/2}VZ^{-1/2})^{-1})
      =tr(Z^{-1}Q(I+t\Lambda)^{-1}Q^T)\\
      =tr(Q^TZ^{-1}Q(I+t\Lambda)^{-1})
      =\sum_{i=1}^{n}(Q^TZ^{-1}Q)_{ii}(1+t\Lambda_i)^{-1}$\\
      where $Z^{-1/2}VZ^{-1/2}=Q\Lambda Q^T$ and actually $h$ is a convex function because the sum of a positive value of $(Q^TZ^{-1}Q)_{ii}$ multiplies $(1+t\Lambda_i)^{-1}$ which is convex.\\
      Above all, $f(x)$ is a convex function.
\end{enumerate}
done.
\end{solution}

\paragraph{Problem 4: Conjugate Function}
~\\
Derive the conjugates of the following functions.
\begin{enumerate}[a)]
    \item $f(x)=\max\{0,1-x\}.$
    \item $f(x)=\ln(1+e^{-x}).$
\end{enumerate}
\begin{solution}
\begin{enumerate}[a)]
  \item \begin{equation}\label{}
          f(x)=\left\{
          \begin{aligned}
          &1-x,x<1\\
          &0,x\ge1\\
          \end{aligned}
        \right.
        \end{equation}
        When $y>0$,$sup(y^Tx-f(x))=+\infty$;\\
        when $y=0$,$sup(y^Tx-f(x))=sup(-f(x))=0$;\\
        when $-1<y<0$,$sup(y^Tx-f(x))=+\infty$;\\
        when $y=-1$,$sup(y^Tx-f(x))=-1$;\\
        when $y<-1$,$sup(y^Tx-f(x))=sup_{x=1}(y^Tx-f(x))=y-f(1)=y$;\\
        so the conjugate of f(x) is $f^*(y)=y,y\in (-\infty,-1)\bigcup\{0\}$.
  \item We have $f'(x)=\frac{-e^{-x}}{1+e^{-x}}$. Because $f'(x)<0$, so $f(x)$ is a decreasing function whose limit is $0$ when $x$ tends to be positive infinity, so when $y>0$,$sup(y^Tx-f(x))=+\infty$;\\
      when $y=0$, $sup(y^Tx-f(x))=sup(-f(x))=0$;\\
      when $y<0$, make $g(x)=yx-\ln(1+e^{-x})$,$g'(x)=y+\frac{e^{-x}}{1+e^{-x}}$, $\frac{e^{-x}}{1+e^{-x}}\in (0,1)$, so when $y\le -1,f'(x)<0$ which means $g(x)$ has no upper bound; when $-1<y<0$, make $f'(x)=0$, we can get $x_0=-\ln\frac{-y}{1+y}$, so $sup(g(x))=g(x_0)=-y\ln\frac{-y}{1+y}+\ln(1+y)$.
      Above all,\begin{equation}\label{}
          f^*(y)=\left\{
          \begin{aligned}
          &0,y=0\\
          &-y\ln\frac{-y}{1+y}+\ln(1+y),-1<y<0\\
          \end{aligned}
        \right.
        \end{equation}
\end{enumerate}
done.
\end{solution}

\paragraph{Problem 5: Optimality Condition}
~\\
Prove that $x^\star=(1,1,-1)$ is optimal for the optimization problem
\begin{gather*}
\begin{matrix}
\text{minimize~~} & (1/2)x^TPx+q^Tx+r\quad~~\\
\text{subject to} & -1\leq x_i\leq1,\quad i=1,2,3
\end{matrix}
\end{gather*}
where
\begin{equation*}
P=\begin{bmatrix}
13&12&-2\\
12&17&~6\\
-2&~6&12\\
\end{bmatrix},\quad\quad q=\begin{bmatrix}
-28.0\\
-23.0\\
~13.0\\
\end{bmatrix},\quad\quad r=1.
\end{equation*}
\begin{solution}
\begin{enumerate}[$\cdot$]
  \item Make $f(x)=\frac{1}{2}x^TPx+q^Tx+r$, the gradient of f(x) is $\nabla f(x)=Px+q$, and $\nabla f_0(x^*)=(-1,0,5)$, because for all y subject to $-1\le y_i\le 1$, $\nabla f_0(x^*)^T(y-x)=-1*(y_1-1)+0*(y_2-1)+5*(y_3+1)=(1-y_1)+5(y_3+1)$. And $-1\le y_i\le 1$, so $(1-y_1)+5(y_3+1)\ge 0$, so $x^*$ is optimal.
\end{enumerate}
done.
\end{solution}


\paragraph{Problem 6: Equivalent Problems}
~\\
Consider a problem of the form
\begin{gather}
\label{quasi}
\begin{matrix}
\text{minimize~~} & f_0(x)/\left(c^Tx+d\right)\quad\quad\quad~\\
\text{subject to} & f_i(x)\leq0,\quad i=1,\dots,m\\
&Ax=b\quad\quad\quad\quad\quad\quad\quad~~
\end{matrix}
\end{gather}
where $f_0,f_1,\dots,f_m$ are convex, and the domain of the objective function is defined as \[\{x\in\textbf{dom } f_0 ~|~c^Tx+d>0\}.\]
\begin{enumerate}[a)]
    \item Show that the problem (\ref{quasi}) is a quasiconvex optimization problem.
    \item Show that the problem (\ref{quasi}) is equivalent to
    \begin{gather}
    \label{convex}
\begin{matrix}
\text{minimize~~} & g_0(y,t)\quad\quad\quad\quad\quad\quad\quad\quad~\\
\text{subject to} & g_i(y,t)\leq0,\quad i=1,\dots,m\\
&Ay=bt\quad\quad\quad\quad\quad\quad\quad\quad\\
&c^Ty+dt=1\quad\quad\quad\quad\quad~~
\end{matrix}
\end{gather}
where $g_i(y,t)=tf_i(y/t)$ and $\textbf{dom }g_i=\{(y,t)~|~y/t\in\textbf{dom }f_i,t>0\}$, for $i=0,1,\dots,m$. The variables are $y\in\mathbf{R}^n$ and $t\in\mathbf{R}$.
  \item Show that the problem (\ref{convex}) is convex.
\begin{solution}
\begin{enumerate}[a)]
  \item Make $g(x)=f_0(x)/(c^Tx+d)$, the domain of $g(x)$ is convex because $f_0$ is convex. Let's see the sublevel sets $S_\alpha=\{x\in\textbf{dom }g|g(x)\le\alpha\}$. From $\frac{f_0(x)}{c^Tx+d}\le\alpha$ and $c^Tx+d>0$ we can get $f_0(x)\le\alpha(c^Tx+d)$, so $S_\alpha$ is convex, therefore, the problem$(6)$ is a quasiconvex optimization problem.
  \item \begin{enumerate}[$\cdot$]
          \item Assume $x$ is feasible for problem$(6)$. Take $t=\frac{1}{c^Tx+d}$,$y=\frac{x}{c^Tx+d}$, let's prove $y,t$ is feasible for problem$(7)$.\\  So $g_0(y,t)=tf_0(y/t)=f_0(x)/(c^Tx+d)$, $g_i(y,t)=tf_i(y/t)=f_i(x)/(c^Tx+d),(i=1,\cdots,m)$, We have\\ $g_i(y,t)\le0$ is equivalent to $f_i(x)\le0$ because $c^Tx+d>0$;\\ $Ay=bt$ is equivalent to $A\frac{x}{c^Tx+d}=b\frac{1}{c^Tx+d}$ which is $Ax=b$;\\
              And $c^Ty+dt=c^T\frac{x}{c^Tx+d}+d\frac{1}{c^Tx+d}=1$.\\
              Therefore, $y,t$ is feasible for problem$(7)$.
          \item On the contrary, assume $y,t$ are feasible for problem$(7)$, let's prove $x$ is feasible for problem$(6)$.\\ make $x=\frac{y}{t}$, because $g_i$ is the perspective function of $f_i$ then we must have $t>0$.\\
              $c^Tx+d=(c^T(y/t)+d)=\frac{c^Ty+dt}{t}=\frac{1}{t}$, so $f_0(x)/(c^Tx+d)=tf_0(y/t)=g_0(y,t)$;\\
              And $f_i(x)=g_i(y,t)/t\le0$ because $g_i(y,t)\le0$;\\
              And $Ax=b$ is equivalent to $A(y/t)=b$ which is $Ay=bt$.\\
              So $x$ is feasible for problem$(6)$.
        \end{enumerate}
        Above all, problem$(6)$ is equivalent to problem$(7)$.
  \item Because $f_i$ is convex, so their perspective function $g_i$ is convex too. And $Ay-bt=0$,$c^Ty+dt-1=0$ are affine, so problem$(7)$ is convex.
\end{enumerate}
done.
\end{solution}
\end{enumerate}


\end{document}
